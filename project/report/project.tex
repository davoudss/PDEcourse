\documentclass[10pt,a4paper,twoside, french]{article}
\addtolength{\textheight}{90pt} \addtolength{\topmargin}{-60pt}
\textwidth 164mm \oddsidemargin -2mm \evensidemargin -2mm
\usepackage[utf8]{inputenc}
\usepackage[T1]{fontenc}
\usepackage[english]{babel}
\usepackage{enumerate}
\usepackage{graphicx} 
\usepackage[left=2.5cm,right=2.5cm,top=3.5cm,bottom=3.5cm]{geometry}
%\usepackage{xcolor,rotating,epic,eepic}
\usepackage[font=small,labelfont=bf]{caption}  %% titre dans les minipage
\usepackage{subcaption} %% les sous figure pour les sous legendes
\usepackage{mathtools}   %%   package mathematique
\usepackage{amssymb}  %% symbole mathematique
\usepackage{fancyhdr}  %% pour pied de page et entete et layout
\usepackage{fourier}  %%  police
\usepackage{titlesec}  %% modifier les section
\usepackage{pgfplots} %% provides tools to generate plots and labeled axes easily
\usepackage{color}
\usepackage{movie15}
\usepackage{animate}

\usepackage{tikz}  %%pour les figures
\usetikzlibrary{shapes,arrows,patterns}
\usetikzlibrary{positioning}

\usepackage[colorlinks = true,
            linkcolor = blue,
            urlcolor  = blue,
            citecolor = black,
            anchorcolor = blue]{hyperref} %% permet de renvoyer à la section en cliquant sur la table des matieres
            
            
            
%ecriture des algo
\usepackage{listings}
\lstset{
	language= Matlab,  %choix du language
    frame=tb, % tb=draw a frame at the top and bottom  - single=around
    tabsize=4, % tab space width
    breaklines=true,
    showstringspaces=false, % don't mark spaces in strings
    backgroundcolor=\color{gray!10},
    numbers=none, % display line numbers on the left
%    numbersep=10pt,                   % how far the line-numbers are from the code
%    numberstyle=\normal\color{black}, % the style that is used for the line-numbers
    commentstyle=\color{red}, % comment color
    keywordstyle=\color{blue}, % keyword color
    stringstyle=\color{green}, % string color
    xleftmargin=17pt,
    framexleftmargin=17pt,
    framexrightmargin=0pt,
    framexbottommargin=5pt,
    framextopmargin=5pt,
    escapeinside={(*}{*)}
}
\DeclareCaptionFormat{listing}{\hrulefill\par\vskip1pt#1#2#3}
\captionsetup[lstlisting]{format=listing,singlelinecheck=false, margin=0pt, font={sf},labelfont=bf}
%font={sf}  labelsep=space   %option captionsetup, ,\rule{\dimexpr\textwidth\relax}{0.4pt}

\renewcommand\lstlistingname{Algorithm}  % CHANGER NOM LISTING EN ALGORITHM
%%%%%%%%  Style subsection  %%%%%%%%%%
\usepackage{titlesec}  %% modifier les section
\titleformat*{\section}{\bf}
\titleformat*{\subsection}{\rm}

\numberwithin{equation}{section}
\numberwithin{figure}{section}
\numberwithin{table}{section}

%%%%%%%%%%%  COMMANDE  %%%%%%%%%%%%%%%%%%%%%%%
\newcommand{\vect}[1]{\mathbf{#1}}
\newcommand{\eq}{\Longleftrightarrow}
\newcommand{\chap}[1]{\widehat{#1}}
\newcommand{\bn}{\mathbf{n}}
\newcommand{\bx}{\mathbf{x}}
\newcommand{\R}{\mathbb{R}}
\newcommand{\N}{\mathbb{N}}
\newcommand{\Po}{\mathbb{P}}
\newcommand{\EspAp}[1]{\text{\textbf{#1}}_h}
\newcommand{\enstq}[2]{\left\{#1\mathrel{}\middle|\mathrel{}#2\right\}}
\newcommand{\nm}{|\!|\!|}
\newcommand{\norme}[1]{\left\Vert #1\right\Vert}
\renewcommand{\div}[1]{\text{div } \vect{#1}}
\newcommand{\rot}[1]{\text{\textbf{rot} } \vect{#1}}
\newcommand{\grad}[1]{\text{\textbf{grad} } \vect{#1}}
\newcommand{\diff}{\mathop{ }\mathopen{ }\mathrm{d}}
\newcommand{\restreinta}[1]{\mathclose{}|\mathopen{}_{#1}}
\newcommand{\prodscal}[2]{\left( #1\ ,\ #2\right)}


\begin{document}

\begin{figure}[h]
\centering
\includegraphics[scale=.5]{fig/kth}
\vspace{-1.5cm}
\end{figure}
	\vspace{1cm}    
    \begin{center}
       		\rule{10cm}{1pt} \\[0.6cm]         %% ligne horizontale   \rule[raise-height]{width}{thickness}  
        	{\huge Project \\[0.2cm]
        	\Large An embedded boundary method for the wave \\
        	equation with discontinuous coefficients \\[0.2cm]
         \Large  Numerical analysis for PDE's\\[0.2cm] 
          \large Thomas Frachon - Davoud Saffar Shamshirgar}  \\[0.6cm]
    		\rule{10cm}{1pt} \\[0.5cm]  
	\end{center}    
	
\section{introduction}
The main goal of this manuscript is two present a second order accurate embedded boundary method for a two dimensional wave equation with discontinuous coefficients. The method is based on the Finite difference method with some special treatments to handle the discontinuity at the interface. The method can be used together with the Dirichlet and Neumann boundary conditions. The method discretizes the second order hyperbolic equation directly. Any second order hyperbolic equations can be reformulated to a system of first order PDEs. Different methods have been proposed to treat these systems such as the $h$-Box method by Berger et. al.\ \cite{berger} in the first order formulations. For linear wave propagations, an staggered grid is used, however these type of grids cannot be used simply for any complex boundaries that intersect the grid arbitrarily. 

There exist other methods based on unstructured grids such as Finite volume or Discontinuous Galerkin method to solve the wave equation with jumps in the boundary. Other authors have considered immersed finite element method which is based on the unfitted mesh.

\section{Outline of the algorithm}
We consider a scalar second order wave equation in a two dimensional domain (see figure \ref{fig:domain}) with a piecewise constant coefficient,
\begin{align*}
\rho(\bx)=\left\lbrace
\begin{array}{cc}
\rho_I, & \bx\in\Omega_I, \\
\rho_{II}, & \bx\in\Omega_{II}.
\end{array}
\right.
\end{align*}
On each subdomain $\Omega_I$ and $\Omega_{II}$, we consider the solutions $u(\bx,t)$ and $w(\bx,t)$ respectively satisfying the following equations
\begin{align}
u_{tt} &= \dfrac{1}{\rho_I}\Delta u+F(\bx,t), \quad \bx\in\Omega_I,\quad t\geq 0,\\
w_{tt} &= \dfrac{1}{\rho_{II}}\Delta w+F(\bx,t), \quad \bx\in\Omega_{I}I,\quad t\geq 0.
\end{align}
and the initial conditions
\begin{align}
u(\bx,0) &= U_0(\bx),\quad ~u_t(\bx,0)=U_1(\bx), \bx\in\Omega_I, \\
w(\bx,0) &= W_0(\bx),\quad w_t(\bx,0)=W_1(\bx), \bx\in\Omega_{II},
\end{align}
where $F(\bx,t)$ is a forcing term. Now assume a smooth interface $\Gamma$ between the two subdomains such that the solutions $u$ and $w$ match by two jump conditions
\begin{center}
\begin{minipage}[c]{.5\textwidth}
\begin{align}
u &= w, &\bx\in\Gamma, \quad t\geq 0, \label{eq:jump_1} \\
\dfrac{1}{\rho_I}\dfrac{\partial u}{\partial n} &= -\dfrac{1}{\rho_{II}}\dfrac{\partial w}{\partial n}, &\bx\in\Gamma, \quad t\geq 0.
\label{eq:jump_2}
\end{align}
\end{minipage}
\end{center}

\begin{center}
\vspace{.5cm}
\begin{minipage}[c]{0.5\textwidth}
\centering
\begin{tikzpicture}
\newcommand{\E}{(-3, -2) rectangle (3,2)}
\newcommand{\A}{(0,0) circle (1.5)}

\draw[fill=white] \E;
\draw[pattern = north west lines] \A;
\draw (-2,-1) node[below left] {$\Omega_{II}$};
\draw (0,-0.5) node[below left] {$\Omega_I$};

\end{tikzpicture}
\end{minipage}
\captionof{figure}{The Domain is divided into two subdomains $\Omega_I$ and $\Omega_{II}$.}
\label{fig:domain}
\end{center}

The Laplacian is discretized using a second order accurate approximation given by 
\begin{align}
\Delta_h u_{i,j}^n := \dfrac{1}{h^2}\left( u_{i\pm1,j}^n+u_{i,j\pm1}^n-4u_{i,j}^n\right), \quad \bx_{i,j}\in\Omega_I.
\label{eq:laplace}
\end{align}
A similar formula can be written for the solution on subdomain $\Omega_{II}$. Moreover we need to define a set of ghost points for the interior subdomain $\Omega_{I}$, (see figure \ref{fig:ghost})
\begin{align}
G_I = \lbrace (i,j), \bx_{i,j}\notin \Omega_I, \text{ but at least one of } \bx_{i\pm1,j}\in\Omega_I \text{ or } \bx_{i,j\pm1}\in\Omega_I\rbrace.
\end{align}
The corresponding expression can be used for the ghost points in $\Omega_{II}$. Note that the solution at the ghost points are obtained using the jump conditions \eqref{eq:jump_1} and \eqref{eq:jump_2}. We can derive a second order approximation for the value and the derivative at the interface using four Lagrange interpolations (see figure 1 in \cite{main_paper}).

\begin{center}
\vspace{.5cm}
\begin{minipage}[c]{0.4\textwidth}
\begin{tikzpicture}[scale = .8]
\newcommand{\E}{(-4, -3) rectangle (4,3)}
\newcommand{\A}{(0,0) circle (1.5)}

\draw[fill=white] \E;
\draw[fill=white] \A;
\draw (-4,-3) grid (4,3);
\draw (-2,0)  node{\LARGE $\bullet$};
\draw (-2,1)  node{\LARGE $\bullet$};
\draw (-2,-1)  node{\LARGE $\bullet$};
\draw (-1,2)  node{\LARGE $\bullet$};
\draw (-1,-2)  node{\LARGE $\bullet$};
\draw (0,2)  node{\LARGE $\bullet$};
\draw (0,-2)  node{\LARGE $\bullet$};
\draw (1,2)  node{\LARGE $\bullet$};
\draw (1,-2)  node{\LARGE $\bullet$};
\draw (2,0)  node{\LARGE $\bullet$};
\draw (2,1)  node{\LARGE $\bullet$};
\draw (2,-1)  node{\LARGE $\bullet$};


\draw (-3,-2) node[below left] {$\Omega_{II}$};
\draw (0,-0.5) node[below left] {$\Omega_I$};

\end{tikzpicture}
\end{minipage}
\captionof{figure}{The Domain is divided into two subdomains $\Omega_I$ and $\Omega_{II}$.}
\label{fig:ghost}
\end{center}

Consider the following one dimensional wave equation with discontinuous wave propagation speed. 
\begin{center}
\begin{minipage}[c]{.7\textwidth}
\begin{align}
u_{tt} &= u_{xx}, \quad &x_{\min}&\leq x\leq 0, \quad &t\geq0,\\
w_{tt} &= c^2w_{xx}, \quad &0&\leq x\leq x_{\max}, \quad &t\geq0,\\
u(x,0) = U_0(x), \quad u_t(x,0) &= U_1(x), \quad &x_{\min}&\leq x\leq 0,\\
w(x,0) = W_0(x), \quad w_t(x,0) &= W_1(x), \quad &0&\leq x\leq x_{\max}.
\label{eq:1d_equations}
\end{align}
\end{minipage}
\end{center}

where $x_{\min}<0$ and $x_{\max}>0$, subject to the Dirichlet boundary conditions
\begin{align}
u(x_{\min},t) = \sin(\pi t), \quad w(x_{\max},t) = 0, \quad t\geq 0.
\label{eq:1d_bc_1}
\end{align}
We also implemented the method using a Dirichlet boundary condition at $x_{\min}$ and Homogeneous Neumann boundary condition at $x_{\max}$. 
\begin{align}
u(x_{\min},t) = \sin(\pi t), \quad w_x(x_{\max},t) = 0, \quad t\geq 0.
\label{eq:1d_bc_2}
\end{align}
The jump conditions at the interface are
\begin{align}
u(0,t) &= w(0,t), \nonumber \\
u_x(0,t) &= c^2w_x(0,t).
\label{eq:1d_jump}
\end{align}
Also we shift the grid to make sure that the discontinuity is not at one of the grid points. As in \eqref{eq:laplace} we discretize the system of equations in \eqref{eq:1d_equations} and the  boundary conditions.

At the interface we compute the value and the derivative using Lagrange interpolations (see figure \ref{fig:1d_interp}). Next, we discretize the jump condition and using the discretized system of equations and simplifications, the semi- discrete problem can be written in a matrix form as 
$$Bw_{tt}=\frac{1}{h^2}Aw,$$
where $A$ and $B$ are tridiagonal matrices. To solve this second order ODE, we can convert the system into two 1st order ODEs 
\begin{align}
C_1 Z_t = C_2 Z, \quad Z = [v, v_t]^T,
\end{align}
and use implicit Euler method to iterate in time
\begin{align}
Z^{n+1} = (I-C_1^{-1}C_2dt)^{-1}Z^n.
\end{align} 


{
\centering
\vspace{.5cm}
\begin{minipage}[c]{0.7\textwidth}
\begin{tikzpicture}[scale = 1]
\draw [thick, ->] (-5,0) -- (6.6,0);
\foreach \x  in {-2,...,1}{ 
   \draw (2*\x cm, 2pt) -- (2*\x cm, -2pt) node[anchor = north, yshift=-3ex] {$u_{\x }$};
   }
\foreach \x  in {0,...,3}{ 
   \draw (2*\x cm, 2pt) -- (2*\x cm, -2pt) node[anchor = south, yshift=+3ex] {$w_{\x }$};   
   }
\draw [thick,dashed] (0.8,0.7) -- (0.8,-0.7);
\draw (0.8,1) node {$x=0$}; 
\draw (0.4,0.2) node {$\alpha h$}; 
\draw (1.4,0.2) node {$(1-\alpha) h$}; 

\end{tikzpicture}
\end{minipage}
\captionof{figure}{The interpolation procedure at the interface.}
\label{fig:1d_interp}
}

\newpage
\section{Results}
Here we provide the solution of the system in \eqref{eq:1d_equations} with boundary conditions \eqref{eq:1d_bc_1} and \eqref{eq:1d_bc_2} and different wave speeds $c=0.5,1,2$. 

\begin{figure}[ht!]
\centering
\includemovie{3cm}{3cm}{fig/final_wave1d_dirichlet_c5.gif}
\includemovie{3cm}{3cm}{fig/final_wave1d_dirichlet_c10.gif}
\includemovie{3cm}{3cm}{fig/final_wave1d_dirichlet_c20.gif}
\caption{The solution of the 1d wave equation with discontinuity and Dirichlet boundary condition at both ends \eqref{eq:1d_bc_1}. (left) $c=0.5$, (middle) $c=1$, (right) $c=2$.}
\end{figure}

\begin{figure}[ht!]
\centering
\includemovie{3cm}{3cm}{fig/final_wave1d_neumann_c5.gif}
\includemovie{3cm}{3cm}{fig/final_wave1d_neumann_c10.gif}
\includemovie{3cm}{3cm}{fig/final_wave1d_neumann_c20.gif}
\caption{The solution of the 1d wave equation with discontinuity and Dirichlet boundary condition at $x_{\min}$ and Neumann boundary condition at $x_{\max}$ \eqref{eq:1d_bc_2}. (left) $c=0.5$, (middle) $c=1$, (right) $c=2$.}
\end{figure}

\begin{thebibliography}{0}

\bibitem{main_paper}
H-.O., Kreiss, and N.A. Petersson,
\emph{An embedded boundary method for the wave equation with discontinuous coefficients},
SIAM J. Sci. Comput., 28 (2006), No. 6, pp. 2054--2074.

\bibitem{berger} 
M.J., Berger, C. Helzel, R.J., LeVeque,
\emph{h-Box methods for the approximations of hyperbolic conservation laws on irregular grids,}
SIAM J. Numer. Anal., 41 (2003), pp. 893--918.


\end{thebibliography}	
	
\end{document}	
